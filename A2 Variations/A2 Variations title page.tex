% !TEX TS-program = pdflatex
% !TEX encoding = UTF-8 Unicode

% This is a simple template for a LaTeX document using the "article" class.
% See "book", "report", "letter" for other types of document.

\documentclass[12pt]{article} % use larger type; default would be 10pt

\usepackage[utf8]{inputenc} % set input encoding (not needed with XeLaTeX)

%%% Examples of Article customizations
% These packages are optional, depending whether you want the features they provide.
% See the LaTeX Companion or other references for full information.

%%% PAGE DIMENSIONS
\usepackage{geometry} % to change the page dimensions
\geometry{a4paper} % or letterpaper (US) or a5paper or....
% \geometry{margin=2in} % for example, change the margins to 2 inches all round
% \geometry{landscape} % set up the page for landscape
%   read geometry.pdf for detailed page layout information

\usepackage{graphicx} % support the \includegraphics command and options

% \usepackage[parfill]{parskip} % Activate to begin paragraphs with an empty line rather than an indent

%%% PACKAGES
\usepackage{booktabs} % for much better looking tables
\usepackage{array} % for better arrays (eg matrices) in maths
\usepackage{paralist} % very flexible & customisable lists (eg. enumerate/itemize, etc.)
\usepackage{verbatim} % adds environment for commenting out blocks of text & for better verbatim
\usepackage{subfig} % make it possible to include more than one captioned figure/table in a single float
% These packages are all incorporated in the memoir class to one degree or another...

%%% HEADERS & FOOTERS
\usepackage{fancyhdr} % This should be set AFTER setting up the page geometry
\pagestyle{fancy} % options: empty , plain , fancy
\renewcommand{\headrulewidth}{0pt} % customise the layout...
\lhead{}\chead{}\rhead{}
\lfoot{}\cfoot{\thepage}\rfoot{}

%%% SECTION TITLE APPEARANCE
\usepackage{sectsty}
\allsectionsfont{\sffamily\mdseries\upshape} % (See the fntguide.pdf for font help)
% (This matches ConTeXt defaults)

%%% ToC (table of contents) APPEARANCE
\usepackage[nottoc,notlof,notlot]{tocbibind} % Put the bibliography in the ToC
\usepackage[titles,subfigure]{tocloft} % Alter the style of the Table of Contents
\renewcommand{\cftsecfont}{\rmfamily\mdseries\upshape}
\renewcommand{\cftsecpagefont}{\rmfamily\mdseries\upshape} % No bold!

%%% END Article customizations

%%% The "real" document content comes below...

\title{Variations on the Interval of a Fourth}
\author{J. Monks - 7235 (centre 40109)}
\date{} % Activate to display a given date or no date (if empty),
         % otherwise the current date is printed 

\begin{document}
\maketitle
\pagenumbering{gobble}


\vspace{3 cm}

\section*{Notes}

`Variations on the Interval of a Fourth' is, as the title suggests, strongly based on one interval. The theme begins with a fourth, from C to F, and contains other references to it in the subsequent bars. Variation I has the melody in the left hand, with sprightly quaver movement and occasional triplet figures accompanied by short chords highlighting the harmony in the right hand. The motif of a fourth is more clearly featured in Variation II, where every other chord is one consisting of a B$\flat$, E$\flat$, and A$\flat$. Variation III is more relaxed, providing relief from the cacophony of the previous section. It realises the fourth by the movement of the bass between F and B$\flat$, C and F, and D and G. The variation ends with a form of bridge passage which links to Variation IV. This variation is explicitly bitonal from the outset, which is shown by the differing key signatures. The upper staff is in B, and the lower in B$\flat$. The fourth is shown firstly in the left hand with the repeated movement from B$\flat$ to F to C, and in the third bar, E$\flat$ to B$\flat$ to F to C. The first six bars are then repeated with a bass of E$\flat$, a fourth higher. The first twelve bars are then repeated, this time with the substitution of an arpeggioic pattern of fourths every third bar. This leads in to Variation V, in which there is a constant arpeggio of fourths underlining the melody, which is shared between the hands. Bar 103 sees a recapitulation of the melody from the theme, this time in B$\flat$ and in a rather different harmonic and textural setting. This variation ends with extensive arpeggios of fourths up and down the keyboard, which lead in to the coda. The coda uses chromatic movement and a phrase with cross-rhythms to a bass of C to modulate back to F for the final few bars. The last reference to the interval of a fourth is in the upper part of the wide chords in the penultimate bar.


\end{document}
